% INTRODUÇÃO-------------------------------------------------------------------

\chapter{INTRODUÇÃO}
\label{chap:introducao}

Arquiteturas Heterogêneas são compostas pelo uso de diferentes tipos de componentes em um mesmo nó computacional.
Geralmente essas arquiteturas são compostas por CPUs (\textit{Central Processing Unit}) e aceleradores, como por exemplo, a GPU (\textit{Graphics Processing Unit}).
Atualmente, muitos fabricantes de chips estão integrando os aceleradores junto com a CPU em um mesmo chip, um exemplo disso é o \textit{Intel Graphics}. 

Utilizar aceleradores como a GPU, junto com a CPU aumenta o desempenho e diminui o gasto energético, porém,
o desenvolvimento de aplicações paralelas para arquiteturas heterogêneas é de certa forma visto como algo desafiador.
Para o desenvolvedor conseguir aproveitar totalmente os componentes e recursos, o mesmo deverá identificar e distribuir as instruções mais adequadas em alguns destes componentes,
para então tentar alcançar o melhor desempenho da aplicação.
Atualmente existem ambientes que se encarregam desta distribuição das instruções para os componentes da arquitetura, como por exemplo, o StarPU.

Neste contexto, o trabalho proposto avaliará o desempenho de uma aplicação quando executada em um destes ambientes.
Para conseguir explorar os ambientes de execução em arquiteturas heterogêneas, será necessário desenvolver uma aplicação.
Essa aplicação será a de uma simulação de transferência de calor em uma placa metálica bidimensional, utilizando o paradigma de paralelismo de tarefas.

Nesse paradigma, o código da aplicação registra a criação de tarefas e a dependência de dados que existe entre elas.
Esse registro de criação de tarefas é direcionado a um ambiente de execução, que se encarrega de distribuir as tarefas de maneira igualitária entre os recursos computacionais.

Em um primeiro momento será implementado uma versão sequencial da aplicação e posteriormente, será implementado uma versão paralela, para ser executada no ambiente \textit{StarPU}.
Ao final desse trabalho, serão coletados dados sobre as duas implementações e realizado uma análise de desempenho, com o objetivo de avaliar se a aplicação utilizando o paradigma de paralelismo de tarefas, terá maior desempenho quando executada no ambiente StarPU.

O presente trabalho está estruturado da seguinte maneira: a seção 2 apresenta os objetivos gerais e específicos, a sessão 3 a justificativa do projeto de pesquisa e a seção 4 o referencial teórico.
Em seguida a seção 5 apresenta a metodologia e a sessão 6 o cronograma de atividades.
Por fim a seção 7 apresenta os resultados esperados.