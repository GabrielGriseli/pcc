% INTRODUÇÃO-------------------------------------------------------------------

\chapter{INTRODUÇÃO}
\label{chap:introducao}

Arquiteturas Heterogêneas são compostas pelo uso de diferentes tipos de componentes em um mesmo nó computacional.
Geralmente essas arquiteturas são compostas por CPUs (\textit{Central Processing Unit}) e aceleradores, como por exemplo, a GPU (\textit{Graphics Processing Unit}).
Atualmente, muitos fabricantes de chips estão integrando a CPU junto com aceleradores em um mesmo chip, como por exemplo o \textit{Intel Graphics}. 

Por mais que esse tipo de arquitetura aumente o desempenho e diminua o gasto energético,
o desenvolvimento de aplicações paralelas para arquiteturas heterogêneas é de certa forma visto como algo desafiador.
Para o desenvolvedor conseguir aproveitar totalmente os componentes e recursos, o mesmo deverá identificar e distribuir as instruções mais adequadas em alguns destes componentes,
para então tentar alcançar o melhor desempenho da aplicação.
Atualmente existem ambientes que se encarregam desta distribuição das instruções para os componentes da arquitetura, como por exemplo, o \textit{StarPU}.

Neste contexto, o trabalho proposto explorará a programação de aplicações paralelas baseadas em tarefas, utilizando o paradigma de Grafo de Dependência de Tarefas.
Nesse paradigma, o código da aplicação registra a criação de tarefas e a dependência de dados que existe entre elas.
Esse registro de criação de tarefas é direcionado a um ambiente de execução, que se encarrega de distribuir as tarefas de maneira igualitária entre os recursos computacionais.

Como aplicação, será desenvolvido uma simulação de transferência de calor (decomposição cartesiana) em uma placa metálica bidimensional de forma sequencial.
Posteriormente, será implementado uma versão paralela da aplicação, para ser executada no ambiente \texit{StarPU}.
Ao final desse trabalho, será coletado dados sobre as duas implementações e realizado uma análise de desempenho, para então, tirar as conclusões sobre o trabalho.

O presente trabalho está estruturado da seguinte maneira: a seção 2 apresenta os objetivos gerais e específicos, a sessão 3 a justificativa do projeto de pesquisa e a seção 4
o referencial teórico.
Em seguida a seção 5 apresenta a metodologia e a sessão 6 o cronograma de atividades.
Por fim a seção 7 apresenta os resultados esperados.