% JUSTIFICATIVA-------------------------------------------------------------------

\chapter{JUSTIFICATIVA}
\label{chap:justificativa}

A maioria dos sistemas de processamento de alto desempenho tem atualmente a sua arquitetura heterogênea,
composta por CPUs (\textit{Central Processing Unit}) \textit{multicore} e aceleradores como a GPU (\textit{Graphics Processing Unit}).
O desenvolvimento de aplicações para essas arquiteturas é feito através de APIs (\textit{Application Programming Interface}), ou bibliotecas.
Essas APIs são específicas para cada tipo de componente de uma arquitetura heterogênea, por exemplo, nas aplicações que exploram os recursos
das CPUs \textit{multicore} pode se usar a API OpenMP (\textit{Open Multi-Processing}) \cite{openmp:2018} ou a biblioteca
Intel TBB \cite{inteltbb:2018} (\textit{Threading Building Blocks}).
Já nas aplicações que visão utilizar a GPU pode se usar OpenCL(\textit{Open Computing Language}) \cite{opencl:2018} ou CUDA (\textit{Compute Unified Device Architecture}) \cite{cuda:2018}.


De acordo com \cite{intrArqHete:2012}, caso o desenvolvedor queira extrair o melhor desempenho possível e utilizar ambos recursos de uma
arquitetura heterogênea, é necessário que o mesmo saiba identificar quais são as tarefas mais adequadas para a execução em um acelerador
e aquelas mais adequadas à execução em uma CPU \textit{multicore}.