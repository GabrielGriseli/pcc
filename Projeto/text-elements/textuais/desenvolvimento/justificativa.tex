% JUSTIFICATIVA-------------------------------------------------------------------

\chapter{JUSTIFICATIVA}
\label{chap:justificativa}

A grande maioria dos sistemas de processamento de alto desempenho tem a sua arquitetura Heterogênea, o desenvolvimento de aplicações para esse tipo de arquitetura é feito através de APIs ou bibliotecas. Essas APIs são específicas para cada tipo de componente de uma arquitetura Heterogênea, por exemplo, nas aplicações que exploram os recursos das CPUs multicores pode se usar a API OpenMP. Já nas aplicações que visão utilizar a GPU pode se usar CUDA ou OpenCL.

Caso o programador queira extrair o melhor desempenho possível e utilizar ambos recursos dessa arquitetura, o mesmo devera utilizar diferentes APIs para isso. Outra questão que é observada, é o problema da portabilidade quando é desenvolvido uma aplicação que visa usar a GPU como recurso, pois a maioria das instruções são de baixo nível, fazendo com que uma aplicação fique restrita a um determinado modelo e/ou fabricando de hardware.

Atualmente existem ambientes que se encarregam da distribuição igualitária das instruções para os recursos da arquitetura, visando um melhor desempenho das aplicações que serão executadas em uma arquitetura heterogênea. A utilização desses ambientes facilita o desenvolvimento das aplicações para esse tipo de arquitetura.