% JUSTIFICATIVA-------------------------------------------------------------------

\chapter{JUSTIFICATIVA}
\label{chap:justificativa}

A maioria dos sistemas de processamento de alto desempenho tem atualmente a sua arquitetura heterogênea,
composta por CPUs (\textit{Central Processing Unit}) \textit{multicore} e aceleradores como a GPU (\textit{Graphics Processing Unit}).
O desenvolvimento de aplicações para essas arquiteturas é feito através de APIs (\textit{Application Programming Interface}), ou bibliotecas.
Essas APIs são específicas para cada tipo de componente de uma arquitetura heterogênea. Por exemplo, nas aplicações que exploram os recursos
das CPUs \textit{multicore} pode se usar a API OpenMP (\textit{Open Multi-Processing}) \cite{openmp:2018} ou a biblioteca
Intel TBB \cite{inteltbb:2018} (\textit{Threading Building Blocks}).
Já nas aplicações que visam utilizar a GPU pode se usar OpenCL(\textit{Open Computing Language}) \cite{opencl:2018} ou CUDA (\textit{Compute Unified Device Architecture}) \cite{cuda:2018}.

De acordo com \cite{intrArqHete:2012}, caso o desenvolvedor queira extrair o melhor desempenho possível e utilizar ambos recursos de uma
arquitetura heterogênea, é necessário que o mesmo saiba identificar quais são as tarefas mais adequadas para a execução em um acelerador
e aquelas mais adequadas à execução em uma CPU \textit{multicore}.

Outra questão que é observada é sobre a portabilidade de um código para uma arquitetura heterogênea.
Segundo \cite{problemsArqHete:2013}, um dos grandes desafios para utilizar os recursos heterogêneos é a diversidade de dispositivos em diferentes máquinas.
Pois um programa que tenha sido otimizado para um determinado acelerador, pode não funcionar tão bem na próxima geração de processadores ou em um dispositivo de um outro fornecedor.
No caso da GPU a programação é feita em um nível próximo ao do \textit{hardware}, fazendo com que muitas vezes o código fique restrito a um modelo e ou fabricante \cite{pinto2011ambientes}.

Vale salientar que o desempenho das CPUs e GPUs também variam entre as máquinas.
Uma execução específica pode funcionar melhor na CPU, enquando que em outra máquina pode funcionar melhor da GPU.
Existe também alguns problemas no escalonamento, se uma aplicação tem desempenho melhor na GPU, mas esta mesma está sobrecarregada e a CPU estiver ociosa, pode ser necessário balancear
a carga de trabalho entre os dois recursos \cite{problemsArqHete:2013}.

Atualmente existem ambientes de execução que dão suporte a CPUs e GPU simultaneamente.
Esses ambientes se encarregam de distribuir as tarefas de maneira igualitária entre os recursos computacionais CPUs e GPUs.
Um fator importante é que esses ambientes implementam escalonadores que são responsáveis pelo desempenho final da aplicação, libertando o desenvolvedor da necessidade de adaptar
a aplicação especificamente à máquina destino e unidades de processamento \cite{kumar:tel-01538516}.

É importante também dizer que a utilização destes ambientes mantêm o código portável entre diferentes modelos e ou fabricantes de hardware  \apud{dolbeau2007hmpp}{pinto2011ambientes}