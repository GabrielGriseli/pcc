% METODOLOGIA------------------------------------------------------------------

\chapter{METODOLOGIA}
\label{chap:metodologia}

Para conseguir explorar os ambientes de execução em arquiteturas heterogêneas é necessário desenvolver uma aplicação. 
Essa aplicação desenvolvida, será a de uma simulação de transferência de calor em uma placa metálica bidimensional.

Inicialmente, será feita uma pesquisa sobre arquiteturas heterogêneas e como funcionam os ambientes de execução para estas arquiteturas,
mais especificamente o ambiente StarPU.
Também serão estudados alguns trabalhos relacionados, que utilizam este ambiente e o paradigma de paralelismo de tarefas.

Posterior a isso, será realizado a implementação sequencial da simulação.
Este passo é interessante para entender como funciona a transferência de calor e as dependências de dados que existem na aplicação.

Após a implementação sequencial, será necessário estudar e configurar o ambiente StarPU. Para a configuração do mesmo,
será estudado a documentação do ambiente e implementado alguns \textit{scripts} do próprio tutorial.
O tutorial do StarPU é importante tanto para entender como funciona o ambiente de execução, como para configurar de forma correta o ambiente.

Em seguida, será realizado a implementação paralela da simulação de transferência de calor utilizando o paradigma de paralelismo de tarefas.
Após isso, a aplicação será executada no ambiente StarPU.

Por fim, será verificado qual é a melhor técnica de análise de desempenho para os dados obtidos nas execuções, para então concluir, se aplicação paralela teve melhor desempenho, ou não, quando comparada com a aplicação sequencial.