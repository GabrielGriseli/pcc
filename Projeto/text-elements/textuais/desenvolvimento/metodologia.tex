% METODOLOGIA------------------------------------------------------------------

\chapter{METODOLOGIA}
\label{chap:metodologia}

Para conseguir explorar os ambientes de execução em arquiteturas heterogêneas será necessário desenvolver uma aplicação.
Essa aplicação que será desenvolvida será a de uma simulação de transferência de calor em uma placa metálica bidimensional,
utilizando a decomposição cartesiana.

Inicialmente, será feita uma pesquisa sobre arquiteturas heterogêneas e como funcionam os ambientes de execução para estar arquiteturas,
mais especificamente o ambiente StarPU.
Também serão estudados alguns trabalhos relacionados que utilizam este ambiente e o paradigma de grafo de dependências de tarefas.

Posterior a isso, será realizado a implementação de forma sequencial da simulação.
Este passo é interessante para entender como funciona a transferência de calor utilizando a decomposição cartesiana e as dependências de
dados que existem na aplicação.

Apos a implementação sequencial será necessário estudar e configurar o ambiente StarPU. Para a configuração do mesmo,
será estudado a documentação do ambiente e implementado
alguns scripts do próprio tutorial do StarPU para validar a configuração.

Em seguida, será realizado a implementação paralela da simulação de transferência de calor.
Para isto, será utilizado o paradigma de grafos de dependências de tarefas para então executar no ambiente StarPU.

Por fim, será realizado testes de desempenho sobre as execuções das duas implementações e tirado as conclusões do trabalho.