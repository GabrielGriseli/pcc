% METODOLOGIA------------------------------------------------------------------

\chapter{METODOLOGIA}
\label{chap:metodologia}

Para conseguir explorar os ambientes de execução em arquiteturas heterogêneas, será necessário desenvolver uma aplicação como estudo de caso. Essa aplicação deverá simular o comportamento real de uma aplicação para Computação de Alto Desempenho. 

A aplicação desenvolvida será a de uma simulação de transferência de calor em uma placa metálica bidimensional. A escolha dessa simulação se dá pela grande quantidade de dependência de dados que existem na transferência de calor, sendo possível desenvolvê-la com o paradigma de paralelismo e tarefas e direcioná-la a um ambiente de execução.

Inicialmente, será feita uma pesquisa sobre arquiteturas heterogêneas e como funcionam os ambientes de execução para estas arquiteturas,
mais especificamente o ambiente StarPU.
Também serão estudados alguns trabalhos relacionados, que utilizam este ambiente e o paradigma de paralelismo de tarefas.

Posterior a isso, será realizado a implementação sequencial da simulação.
Este passo é interessante para entender como funciona a transferência de calor e as dependências de dados que existem na aplicação.

Após a implementação sequencial, será necessário estudar e configurar o ambiente StarPU. Para a configuração do mesmo,
será estudado a documentação do ambiente e implementado alguns \textit{scripts} do próprio tutorial.
O tutorial do StarPU é importante tanto para entender como funciona o ambiente de execução, como para configurar de forma correta o ambiente.

Em seguida, será realizado a implementação paralela da simulação de transferência de calor utilizando o paradigma de paralelismo de tarefas.
Após isso, a aplicação será executada no ambiente StarPU. 
A execução do ambiente será feita em uma máquina com um processador Intel core I5 2310 (2.90GHz com 4 núclos) e uma GPU Nvidia 750TI (640 cudas cores).

Por fim, será avaliado o desempenho da aplicação desenvolvida com o paradigma de paralelismo de tarefas com a aplicação sequencial. Para isso serão analisados os tempos de execução das aplicações e a quantidade de recursos que foram utilizados.

Será também avaliado como a arquitetura heterogênea se comportou com a execução da aplicação no ambiente StarPU, analisando se houve algum tipo de overhead na comunicação dos dados.