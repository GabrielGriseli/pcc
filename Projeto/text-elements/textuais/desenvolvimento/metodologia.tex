% METODOLOGIA------------------------------------------------------------------

\chapter{METODOLOGIA}
\label{chap:metodologia}

Para conseguir explorar os ambientes de execução em arquiteturas heterogêneas será necessário a implementação de uma aplicação.
Essa aplicação que será desenvolvida será a de uma simulação de transferência de calor em uma placa metálica bidimensional,
utilizando a decomposição cartesiana.

Primeiramente será necessário a implementação de forma sequencial da simulação.
Esse passo é interessante para entender como o \textit{kernel} implementa a transferência de calor e as suas dependências com os vizinhos
da decomposição cartesiana, assim também como os ajustes dos \textit{timestemps} que a simulação terá.

Apos a implementação sequencial será necessário a configuração do ambiente StarPU.
Para a configuração do mesmo, será estudado a documentação do ambiente e implementado alguns scripts do próprio tutorial do StarPU
para validar a configuração.

Apos a configuração e estudo do ambiente, será buscado trabalhos relacionados para entender como funciona o paradigma de grafo de dependências de tarefas.
Ao final dessa pesquisa, será necessário implementar a versão paralela da simulação para ser executada no ambiente StarPU.

Por fim, será realizado testes de desempenho sobre as execuções das duas implementações e tirado as conclusões do trabalho.