% REFERENCIAL TEORICO----------------------------------------------------------

\chapter{REFERÊNCIAL TEÓRICO}
\label{chap:ref-teorico}

% \section{Computação de Alto Desempenho}
\section{Programação Paralela}

Um programa de processamento paralelo é um único programa que é executado em vários processadores simultaneamente.
A programação paralela pode ser usada para o reduzir o tempo de execução de um programa,
que busca encontrar uma solução para um problema complexo, como por exemplo, problemas voltados as áreas científicas \cite{hennessy2014organizaccao}.

A diferença de um programa sequencial para um programa paralelo, é que um programa sequencial é visto como uma série de instruções
sequenciais que devem ser executadas em um único processador.
Já um programa paralelo, é visto como um conjunto de partes que podem ser resolvidos concorrentemente,
essas partes, são constituídas de instruções sequenciais \cite{hennessy2014organizaccao,tanenbaum20103a}
% \subsection{Classificação e Terminologia}

\section{Arquiteturas Voltadas a Computação Heterogênea}
Muitas das arquiteturas que estão atualmente nos supercomputadores é voltada a computação heterogênea \cite{meuer2014top500}.
O termo heterogênea, descreve que diferentes arquiteturas possam ser usadas em um mesmo nó computacional, como por exemplo, processadores multicore e aceleradores como a GPU e a FPGAs \cite{intrArqHete:2012}.

\subsection{Processadores Multicore}
\subsection{Aceleradores}

% \section{Programação Paralela em Arquiteturas Heterogêneas}
% \subsection{CUDA}
% \subsection{OpenCL}
% \subsection{Intel tbb}
% \subsection{OpenACC}

\section{Ambientes de Execução para Arquiteturas Heterogêneas}
\subsection{StarPU}
\subsection{Xkaapi}
\subsection{StarSs}

\section{Paradigma de Grafo de Dependência de Tarefas}

\section{Transferência de Calor em Placas Metálicas}