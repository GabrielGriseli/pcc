% REFERENCIAL TEORICO----------------------------------------------------------

\chapter{REFERÊNCIAL TEÓRICO}
\label{chap:ref-teorico}

% \section{Computação de Alto Desempenho}
\section{Programação Paralela}
% \subsection{Classificação e Terminologia}

\section{Arquiteturas Paralelas Heterogêneas}
Muitas das arquiteturas que estão atualmente nos supercomputadores é voltada a computação heterogênea.
O termo heterogeneidade, descreve que diferentes arquiteturas possam ser usadas em um mesmo nó computacional, como por exemplo, processadores multicore e aceleradores como a GPU e a FPGAs \cite{intrArqHete:2012}.



\subsection{Processadores Multicore}
\subsection{Aceleradores}

% \section{Programação Paralela em Arquiteturas Heterogêneas}
% \subsection{CUDA}
% \subsection{OpenCL}
% \subsection{Intel tbb}
% \subsection{OpenACC}

\section{Ambientes de Execução para Arquiteturas Heterogêneas}
\subsection{StarPU}
\subsection{Xkaapi}
\subsection{StarSs}

\section{Paradigma de Grafo de Dependência de Tarefas}

\section{Transferência de Calor em Placas Metálicas}