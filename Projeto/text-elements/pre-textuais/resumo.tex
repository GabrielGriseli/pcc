% RESUMO--------------------------------------------------------------------------------

\begin{resumo}[RESUMO]
\begin{SingleSpacing}

% % Não altere esta seção do texto--------------------------------------------------------
% \imprimirautorcitacao. \imprimirtitulo. \imprimirdata. \pageref {LastPage} f. \imprimirprojeto\ – \imprimirprograma, \imprimirinstituicao. \imprimirlocal, \imprimirdata.\\
% %---------------------------------------------------------------------------------------

O desenvolvimento de aplicações paralelas voltadas a arquiteturas heterogêneas pode vir a ser algo desafiador, pois nessas arquiteturas, os componentes são formados por processadores \emph{multi-core} (CPUs) e placas gráficas (GPUs).
O programador então, deverá distribuir as instruções que devem ser executadas em alguns destes componentes, para tentar alcançar o melhor desempenho da aplicação.
Existem atualmente ambientes que se encarregam desta distribuição das instruções para os componentes da arquitetura, como por exemplo, o \emph{StarPU}. 
Tendo em vista isto, este trabalho explorará o paradigma de Grafo de Dependência de Tarefa, aplicando-o a uma aplicação paralela de simulação de Decomposição Cartesiana sobre o ambiente de execução \emph{StarPU}.
Ao final deste trabalho, será coletado dados sobre esta aplicação para então tirar as conclusões sobre o seu desempenho, comparando com a mesma aplicação desenvolvida de forma sequencial.\\

\textbf{Palavras-chave}: Aplicação Paralela. Arquiteturas Heterogêneas. Computação de Alto Desempenho.

\end{SingleSpacing}
\end{resumo}

% OBSERVAÇÕES---------------------------------------------------------------------------
% Altere o texto inserindo o Resumo do seu trabalho.
% Escolha de 3 a 5 palavras ou termos que descrevam bem o seu trabalho
