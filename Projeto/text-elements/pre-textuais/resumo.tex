% RESUMO--------------------------------------------------------------------------------

\begin{resumo}[RESUMO]
\begin{SingleSpacing}

% % Não altere esta seção do texto--------------------------------------------------------
% \imprimirautorcitacao. \imprimirtitulo. \imprimirdata. \pageref {LastPage} f. \imprimirprojeto\ – \imprimirprograma, \imprimirinstituicao. \imprimirlocal, \imprimirdata.\\
% %---------------------------------------------------------------------------------------

O desenvolvimento de aplicações paralelas voltadas a arquiteturas heterogêneas pode vir a ser algo desafiador, pois nessas arquiteturas, os componentes são formados por processadores \emph{multi-core} (CPUs) e placas gráficas (GPUs).
O programador então, deverá distribuir as instruções que devem ser executadas em alguns destes componentes, para tentar alcançar o melhor desempenho da aplicação.
Existem atualmente ambientes que se encarregam desta distribuição das instruções para os componentes da arquitetura, como por exemplo, o StarPU. 
Tendo em vista isto, este trabalho explorará o paradigma de Paralelismo de Tarefas, aplicando-o em uma simulação de Transferência de Calor em uma placa metálica bidimensional e executada no ambiente StarPU.
Ao final deste trabalho, serão coletados dados sobre esta aplicação buscando avaliar seu desempenho neste ambiente, comparando com a versão sequencial da aplicação.\\

\textbf{Palavras-chave}: Aplicação Paralela. Arquiteturas Heterogêneas. Computação de Alto Desempenho.

\end{SingleSpacing}
\end{resumo}